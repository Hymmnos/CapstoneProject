% This is LLNCS.DEM the demonstration file of
% the LaTeX macro package from Springer-Verlag
% for Lecture Notes in Computer Science,
% version 2.3 for LaTeX2e
%
% The infinite quirkiness of TekMaker ....
% ... to get it to fully update, need to hit F1 a few times,
% ... then F11 a few times (for bibliography),
% ... then F1 a bunch more times
% ... .... and then the .pdf catches up to all the edits
% ...	weird that it takes multiple cycles w/ human action ...
% ...
% ...	or maybe this works ..
%...	you can use hotkeys to run pdflatex and bibtex with F6 and F11 on your keyboard
%... 	respectively, so pressing F6 F11 F6 F6 one at a time (and waiting for texmaker to 
% ...	finish each time) you will update everything in your document
% ...
% ... file needed in same directory :
% ...	llncs.cls
% ...	splncs.bst
% ...	pedestrianreferences.bib (bibliography)
%
\documentclass{llncs}
%
% additional packages added to comply wth format recommendations
%
\usepackage{makeidx}  % allows for indexgeneration
\usepackage[strings]{underscore} % allows underscore
\usepackage{float} % position tables inline
\usepackage{placeins}
\usepackage{caption}
\captionsetup[table]{singlelinecheck=false,
					labelfont=bf,
					justification=raggedright,
					labelsep=period}
\usepackage{textcomp}
\usepackage{colortbl}

\begin{document}
\setlength{\parskip}{0pt}
%
%
\title{Pedestrian Safety -- Fundamental to a Walkable City}
%
\author{Patrick McDevitt\inst{1} \and Preeti Swaminathan\inst{1}
\and Joshua Herrera\inst{1} \and Raghuram Srinivas\inst{2}}
%
\institute{Masters of Science in Data Science, Southern Methodist University,\\
Dallas, TX 75275 USA\\
\and
Southern Methodist University,\\
Dallas, TX 75275 USA\\
\email{\{pmcdevitt, pswaminathan, herreraj, rsrinivas\}@smu.edu}}

\maketitle              % typeset the title of the contribution

\begin{abstract}
In this paper, we present a method to identify urban areas with a higher likelihood of pedestrian safety related events. Pedestrian safety related events are pedestrian–-vehicle interactions that result in fatalities, injuries, accidents without injury, or a near--miss between pedestrian and vehicle. The occurrence of these events detracts from the safety of the citizens of the city. To develop a solution to this problem of identifying likely locations of events, we assembled data sets, primarily from the City of Cincinnati, that include safety reports from a 5 year period, geographic information for these events, a citizen survey that identifies pedestrian reported concerns, and a database of all requests for service for any cause in the city. We augmented that data with sports-tracking geolocation movement data of pedestrians, runners, and cyclists. From this assembled data set we completed unsupervised learning using a self--organizing map, excluding the event data. The event data was then mapped into the self-organizing map clusters to identify the statistical likelihood of events in each cluster. The results indicate a statistically significant association between clusters and events. The results identify locations in urban areas for prioritized remediation enabling a proactive approach to improved pedestrian safety.
\end{abstract}
%
\section{Introduction}
%
\emph{An early-morning walk is a blessing for the whole day} -– Henry David Thoreau \cite{thoreau1906writings}. So, begins the choice every day for urban dwellers -– to walk or not to walk –- to have a blessing as proposed by Thoreau, or to assess the daily commute -- as summarized by Jeff Kober \cite{bowen2015zen}: \emph{My intention is to get done with this commute \dots my intention will not be met until I get out of this car} –- as just a rather unpleasant means to get from point A to point B.

A walkable neighborhood is a neighborhood with the following characteristics : a center (either a main street or public space), sufficient population density to support local businesses and public transit, affordable housing, public spaces, streets designed for bicyclists and pedestrians, and schools and workplaces within walking distance for residents \cite{wal2018walkable}. As the modern urban landscape has evolved in the US over the last fifty years, pedestrianism was not often on the list of high priorities for inclusion into the development of urban environments. As a result of this trend, there have been real, and negative, consequences: economically, epidemiologically, and environmentally on the inhabitants of many cities in western developed countries \cite{speck2013walkable}. Economically, we can observe that the percentage of income spent on transportation for working families has doubled, from one-tenth to one-fifth of household earnings from the 1970s to current era \cite{speck2013walkable}. So much so that working families are currently spending more of their budget on transportation than housing. If we consider the health effects of urban living patterns, we observe that people living in less walkable neighborhoods are nearly twice as likely to be obese than people that live in walkable neighborhoods \cite{speck2013walkable}. This statistic, coupled with the fact that Americans now walk the least of any industrialized nation in the world \cite{lee2014suburban} indicate a growing health problem due in part to a lack of physical activity. When constructed on a per-household basis, carbon mapping clearly demonstrates that suburban dwellers generate nearly twice as much carbon-dioxide, the main pollutant contributing to global warming \cite{climatendclimate}, than do urban dwellers due to longer commutes and larger houses \cite{speck2013walkable}.

There is a growing movement in the US and other western nations to promote the concept of walkable cities as healthier places to live - economically, environmentally and physiologically - than the suburban, exurban, drive-till-you-qualify model of modern western development \cite{leyden2003social} \cite{steffen2008worldchanging} \cite{doyle2006active}. As identified in the Toronto Pedestrian Charter \cite{toronto2002toronto} the six principles for building a vital urban pedestrian environment include: accessibility, equity, health and well-being, environmental sustainability, personal and community safety, and community cohesion and vitality. According to the city of Toronto, this is the first such pedestrian bill of rights in the world and promotes the concept that walking is valued for its social, environmental, and economic benefits.

The US is experiencing an increase in the number of pedestrian fatalities, reaching a 25-year high in 2017, with nearly 6,000 fatalities \cite{domonoske2018pedestrian}. Newspaper articles in the Midwest identify fatal occurrences: \cite{kelley2018police} "An uptick in pedestrians being hit by cars in the Cincinnati and Northern Kentucky area has officials sounding the alarm. Three crashes just this week resulted in the death of three pedestrians."

As one avenue of response, the City of Cincinnati has requested citizen input to identify specific areas in the city which are pedestrian safety concerns. The city created a web-site, which launched in Feb-2018 \cite{cvg2018city}, that allows citizens to specifically identify a location on a map, within a distance of several feet of the area of concern and report the nature of the concern in a functional user interface. The city plans to use this community input to prioritize maintenance and improvement resources.

\section{Pedestrian Safety}
%
The subject pedestrian safety is supported by terminology specific to this domain. A collection of the terminology used in this paper is provided in this section. 

Prime measurements used to report pedestrian safety events are fatalities, injuries, and near misses. The statistics in these categories are quoted in number of events and are typically stated on an annualized and per capita basis.

There are a range of severities associated to the outcomes of pedestrian--vehicle accidents. A continuous real valued response variable that accounts for the both the severity and the frequency of events can be established by accounting for this relative severity. We have implemented a response variable that is a multiple of the number of events and the cost of the event. The cost basis that we used is based on average severity costs for 5 levels of events, as established by the National Safety Council \cite{nsc2012estimating} as shown in Table ~\ref{table:eventseverity}.
%
\FloatBarrier
\begin{table}[!h]
\begin{center}
\caption{Event Cost Severity, 2012 NSC}
\label{table:eventseverity}
\begin{tabular}{p{50mm}  p{50mm}}
\hline
\rule{0pt}{12pt}
Severity & Unit Cost (\$)\\[2pt]
\hline
Fatality 					&	4,538,000\\
Incapacitating injury 		&	230,000\\
Non-Incapacitating injury 	&	58,700\\
Possible injury 			&	28,000\\
No injury 					&	2,500\\[2pt]
\hline
\end{tabular}
\end{center}
\end{table}
\FloatBarrier
%
Table ~\ref{table:terminology} below is presented as a primer on pedestrian safety related terminology along with an explanation of the significance of each term in relation to the evaluation presented in this paper.
%
\FloatBarrier
\begin{table}[!ht]
\caption{Pedestrian safety terminology}
\label{table:terminology}
\begin{center}
\begin{tabular}{p{40mm}  p{80mm}}
\hline
\rule{0pt}{12pt}
Attribute & Description\\[2pt]
\hline
U.S. Census Bureau units of measure & The U.S. Census Bureau reports data within geographic units. The census block is the smallest geographic unit used by the Census Bureau. Census blocks are typically bounded by streets or natural features. There is no standard size, either by surface area or population, that is used by the Census Bureau.  The data reported in a  census block is 100 per cent of households reported data. There is no sampling or estimations used in census block reported data. Census blocks are assembled into block groups, and block groups then constitute census tracts \cite{usgovndgeographic}. \\	
Potential for Safety Improvement (PSI)	&	Measures the actual crash cost minus the expected cost of “similar” sites that can be obtained from the crash cost models. In typical usage, an explanatory model using available features is established to predict some measure of cost (e.g., fatality or injury). \cite{ohgov2017} \\	
Vehicle miles of travel	(VMT) &	A method to account for volume of vehicle traffic. The value is the total annual miles of vehicle travel within a specified zone. Values available from the U. S. Dept of Transportation can be aligned with U.S. Census Bureau urbanized areas. VMT is sometimes also characterized on a per capita basis.\cite{usgov2016vmt}	\\	
Hotspot & In this context, hotspots are areas with higher density or frequency of pedestrian related accidents \cite{xie2017analysis}. 	\\	
Regression--to--the mean & Regression to the mean; since traffic fatality events are low volume incidents (on the order of magnitude of 10s for most cities). RTM is a consideration in studies of pedestrian safety because an area in which fatalities occur in one year may not be repeated in the next, even in the absence of implemented changes.  	\cite{xie2017analysis}\\	
Conditional script questionnaire (CSQ) & A survey tool for assessing expected human behavior under alternative situations. In the context of pedestrian safety, a CSQ requests respondents to self--assess their likelihood to ignore the driving code under different scenarios. The CSQ responses are used in establishing sub--populations that pose increased risk for pedestrian safety. The sub--populations can be based on demographic features (e.g., gender, age) or situational features (e.g., late for work) \cite{gaymard2015conditionality}. 	\\	[2pt]
\hline
\end{tabular}
\end{center}	
\end{table}
\FloatBarrier	

The focus of many pedestrian safety studies is the interaction between pedestrians and vehicles. Prior works have created statistical models to determine the likelihood of crashes, given information about the time of day, victim's age, gender, and other features \cite{brude1993models} \cite{lascala2000demographic} \cite{lyon2002pedestrian} \cite{ladron2004forecasting} \cite{pulugurtha2011pedestrian} \cite{ukkusuri2011random}. The study done by Guo \cite{guo2017effect}, et al examined the patterning and structure of road networks as a factor of pedestrian vehicle interactions (PVIs). Zhang et al \cite{zhang2017quantitative} created a statistical model that classified different types of street crossings to determine which type of crossing was the safest, and gain insight to the relationships between the factors that contribute to a PVI. In our study, we addressed the issue of pedestrian safety in regard to PVIs by proactively identifying intersections which are of high risk, as opposed to prior studies which have only identified the contributing factors.


\section{Data Sets}
%
%
CAGIS, Cincinnati Area Geographic Information System launched an online survey from Feb 2018 to April 2018 for pedestrian safety \cite{cvg2018city}. Users login to http://cagisonline.hamilton-co.org/pedsafetysurvey/ . The survey screen provides users with view of the city, with a drop down of neighborhoods, and a list of issue categories. The user then selects a neighborhood, and selects a pre-defined issue type to report, and writes their comment. If another user selects the same location and issue type, comments are appended as additional comments. This gives an idea of number of users having same issue at a particular location. Survey submissions are anonymous.

Additionally, the date of the issue created, typical mode of user (walk, drive, bicycle), and intersection of the location selected are captured. There are 3788 records in our survey dataset with 8 usable columns.  There are missing data and categorical data in incorrect format. These require EDA to clean-up and update missing information. Other than the survey data, we have various sources of data which would be combined to form the dataset for the project. All acquired from the City of Cincinnati are listed in Table ~\ref{table:cincinnatidatasets}. Supplementary data sets from external sources are listed in Table ~\ref{table:supplementarydatasets}. 

\FloatBarrier
\begin{table}
\begin{center}
\caption{Survey data attributes}
\begin{tabular}{l p{80mm}}
\hline
\rule{0pt}{12pt}
Attribute & Description\\[2pt]
\hline
REQUESTID	&	Survey key\\
REQUESTTYP	&	Provides categories for issue	\\
REQUESTDAT	&	Date on which the issue was raised	\\
COMMENTS	&	Issue description	\\
USERTYPE	&	Mode of transport of the user	\\
NEAR_INTER	&	Intersection nearest to location of issues	\\
NEAR_STR	&	Street of issue	\\
STRSEGID	&	ID specific to city of Cincinnati\\
Additional_COMMENTS	&  Additional descriptions for the same issue by different users	\\
SNA_NAME & Neighborhood name	\\[2pt]
\hline
\end{tabular}
\end{center}	
\end{table}
\FloatBarrier	
%
\FloatBarrier
\begin{table}[!h]
\begin{center}
\caption{Cincinnati based data sets}
\label{table:cincinnatidatasets}
\begin{tabular}{ p{.25\textwidth}  p{.30\textwidth}  p{.45\textwidth}}
\hline
\rule{0pt}{12pt}
Data Set
	& Source
	& Evaluation summary\\[2pt]
\hline
Census and demographic
	&https://www.cincinnati-oh.gov/planning/reports-data/census-demographics
	&Contains census data for each neighborhood, split into census tracts\newline
	Data is organized at the neighborhood level, not street level as our final dataset is\\
Cincinnati open data
	&https://data.cincinnati-oh.gov
	& Contains economic, neighborhood safety, and health related data for city of Cincinnati\newline
	Data is not granular\\
Cincinnati pedestrian safety survey data
	&
	& 
	Contains survey data from citizens who have reported problems by using the web-page\newline
	Data is organized at  the street intersection level\newline
	  Data was collected from Feb - Apr 2018\\
Income and house price
	&http://www.city-data.com/nbmaps/neigh-Cincinnati-Ohio.html
	& Contains statistics on age, house prices, income and more\newline
	Data gathered as a collection of public and private sources\newline
	Data organized at both the neighborhood, and census block group level\\
Traffic crash reports
	&https://data.cincinnati-oh.gov/Safer-Streets/Traffic-Crash-Reports-CPD-/rvmt-pkmq
	& Crash details at intersection is available, with details on person ages and injury level \newline
	Provided by the Cincinnati Police Department \\[2pt]
\hline
\end{tabular}
\end{center}
\end{table}
\FloatBarrier
%
\FloatBarrier
\begin{table}[!h]
\begin{center}
\caption{Supplementary data sets}
\label{table:supplementarydatasets}
\begin{tabular}{ p{.25\textwidth}  p{.30\textwidth}  p{.45\textwidth}}
\hline
\rule{0pt}{12pt}
Data Set
	& Source
	& Evaluation summary\\[2pt]
\hline
Google Maps
	& https://developers.google.--\newline
		com/maps 
	& This API gives us granular data on walking and biking - distance, direction and other information\newline
	Latitude and Longitude information for grid areas can be derived using this API\\
OpenStreetMap
	&https://www.openstreet-- \newline
		map.org
	&Similar to Google maps; we are researching ways to implement this into our study\\
Strava
	&
	&Strava is a website and mobile application used by runners and bikers to upload their GPS-tracked activity to the Strava website\newline
	Data can be used to create a heatmap of high activity areas\\
Walk Score\textsuperscript{\tiny\textregistered}
	&https://www.walkscore	\newline
	.com
	&A scoring scale across US which gives an idea of the current walkability of the city\\
Zillow
	&https://www.quandl.com/ \newline
	data/ZILLOW/M26_NFS-Zillow-
	Home-Value-Index-Metro-NF-Sales-\newline
	Cincinnati-OHh
	&
	The Zillow Home Value Index contains monthly time-series of data which represent Zillow's estimation of the median market value of home sales.\\[2pt]
\hline
\end{tabular}
\end{center}
\end{table}
\FloatBarrier
%
\section{Methods and Experiments}
Analysis approach

For this project, our approach consisted of two parts. Part 1 involved extensive EDA, using unsupervised techniques to identify clusters and trends within the data. Part 2 consisted of adding cost of injury as in Table ~\ref{table:eventseverity} to our dataset to perform regression. The dataset was simplified using correlation metrics. Feature importance  of each regression method was tabulated. This approach  aided in quantifying the intensity of accidents and predicting how expensive accidents might be. Models with high R$^{2}$ and low RMSE are  considered  good models.
Additional algorithms were   used in conjunction with above 2 approaches. T-sne was used as a visualization tool to view clustering in 2-dimensional space, and PCA was used to reduce the dimensionality of the data set for regression. Furthermore, natural language processing was used to extract sentiment from the survey data comment field.
The dataset was shuffle split into 80:20, 20\% holdout for cross-validation. 80\% will be split into 80:20 for training and testing. Machine learning and deep learning algorithms were used as listed in Table ~\ref{table:algorithmsandmethods}. Models were compared with each other on different performance metrics. 
%
\FloatBarrier
\begin{table}
\begin{center}
\caption{Algorithms and evaluation methods}
\label{table:algorithmsandmethods}
\begin{tabular}{p{50mm} p{60mm}}
\hline
\rule{0pt}{12pt}
Algorithm	&	Models	\\[2pt]
\hline
Regression 	&	1.        Multiple Regression	\newline
				2.        Support Vector Machine	\newline
				3.        Decision Tree	\\
Unsupervised Learning	&	1.        Clustering	\newline
				\hspace*{5mm} a.        K-means	\newline
				\hspace*{5mm} b.        Hierarchical 	\newline
				2.        Gaussian Mixture Models	\newline
				3.        Neural Networks	\newline
				\hspace*{5mm} a.        Self Organizing Maps	\\
Additional 	&	1.        T-Distributed Stochastic Neighbor Embedding	\newline
				2.        Principal Component Analysis	\newline
				3.        Linear Discriminate Analysis	\newline
				4.        Natural Language Processing	\\[2pt]
\hline
\end{tabular}
\end{center}
\end{table}
\FloatBarrier
%
\section{Results}
%
%
\section{Analysis}
%
%
\section{Ethics}
%
As a means to evaluate compliance with ethical considerations, we use the model of the ACM Code of Ethics  (the « Code »). Within the Code, there are four primary sections, e.g., « General Ethical Principles », « Professional Leadership Principles », etc., with each primary section providing additional subsections for self-assessment compliance to the Code. For each sub-section, we self-scored categorically as either “Y”, “n/a”, or “D“, where “Y” indicates that the work completed for this project rather obviously complies with the Code, “n/a” indicates that that section of the Code is less obviously significant for this project, and “D“ indicates that that section of the Code identifies a potential ethical dilemma that is worthy of additional discussion to demonstrate compliance or at least point out the potential ethical challenge identified from this self-assessment.
The most significant elements that we self-assess as “D“ are : \S 1.2 - Avoid harm; \S 1.4 - Be fair and take action not to discriminate; and \S 3.7 - Recognize and take special care of systems that become integrated into the infrastructure of society.

From these three elements, we consider that the significant question to evaluate is : how may these research findings be interpreted and used ?

The result of this project provides a recommended prioritization for the allocation of municipal resources for the purpose of improving a pedestrian safety problem. Allocation of  public resources is often as much a political challenge as it is a scientific challenge. There is no global objective function that assigns absolute social value to any decision of resource allocation. That is, in fact, the work and challenge of public officials. Within the general framework of public decision making it is recognized that facts, reports, and recommendations which are or were essentially the result of objective research are frequently interpreted in a way that suits the interpreter for their own agenda – in some cases for personal gain – financially or politically. We have to admit for this case, then, that this evaluation is potentially subject to a personally motivated interpretation. The debate about using scientific research to guide public policy is long and continuing. With that recognition, the task falls to us to identify what steps are taken to reduce the risk of unintended uses of this report.

First, the report as written has limited scope for direct application to policy. The recommendations included are applicable to the specific time period and data evaluation associated to the City of Cincinnati. The methods presented here can be widely applied (and in fact, that is the goal of this research), but in current form it would be difficult to justify using these results for direct resource allocation in any other municipality. The model developed here used very specific local experiences – accidents, reported near accidents, local conditions survey, property valuations, social media, and all other elements that contributed to this model are local and specific to the City of Cincinnati and to the current time period. As such, the specific recommendations are not generalizable. The method is generalizable; the specific results may provide indications of what local elements in other municipalities may prove indicative or at least useful for a similar exercise, but in any rational discourse, it would be difficult to extend the specific recommendations from this study to municipalities beyond the extent of this study.

Secondly, this report is submitted to representatives of the Cincinnati City Council and the Department of Safety . By distributing the results to more than one department and to a reasonably broad audience reduces risk of information being used for narrowly scoped  or interests which are not generally aligned with good public discussion, debate, and utilization. Further, the data and methods deployed here were developed based on early and continuing input from representatives of these multiple departments within the City. Thus, early and often participation of multiple stakeholders provided the opportunity to have balanced input to the research, thus improving likelihood of balanced output and utilization. And, by respecting and incorporating the input from multiple perspectives from within the City provides higher likelihood of acceptance (and perhaps adoption) of the resulting recommendations.

Thirdly, within the friendly confines of the City of Cincinnati, there are several on-going initiatives dedicated to improving pedestrian safety. The other initiatives are, in some cases, significantly funded, and are the work product of several departments within the City, predominantly the Department of Safety, that are the prime stakeholders in promoting public safety in the City. These other initiatives are orders of magnitude more significant, both from resource commitment, and for intended impact, than is this study. It is not our aim to minimize the potential impact of the recommendations from this study, but we are cognizant of the relative significance of this study is but one tooth of a small cog in a larger machinery. In any decision making forum for the City, we consider the likelihood of these results having the capability to be used for unintended or inappropriate outcomes to be remote.

It is by this combination of considerations that we assess the ethical considerations associated of this project to be adequately assessed in the spirit of the ACM Code, and that the identified risks have been appropriately mitigated.
%
\section{Conclusions (and Future Work)}
%
% ---- Bibliography ----
%
\bibliographystyle{splncs}
\bibliography{pedestrianreferences}

\end{document}